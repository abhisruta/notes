\documentclass[11pt, a4paper, abstract=true]{scrartcl}
\usepackage[sexy]{evan}
\usepackage{float}
\usepackage[margin=1.00in]{geometry}
\usepackage{multirow}
\usepackage{longtable}
\usepackage{chemformula}
\def\arraystretch{1.20}

% \clearpairofpagestyles
\setkomafont{pagenumber}{\itshape}
\KOMAoptions{}
\ohead{\footnotesize \textbf{\leftmark}}
\ihead{\footnotesize \textsc{Lab Report II}}
\cfoot{\pagemark}
\begin{document}
\subject{
    CH1202: Lab Report II
}
\title{
    \huge Determination of \\
Degree of Hydrolysis and \\ Hydrolysis Constant by Potentiometry
}
\author{
    Abhisruta Maity \\
    {\normalsize 21MS006}
    \plusemail{am21ms006@iiserkol.ac.in}
}
\date{}
\publishers{
    \normalsize \emph{Indian Institute of Science Education and Research, Kolkata \\
    Mohanpur, West Bengal, 741246, India}
}
\maketitle

\tableofcontents

\section{Aim}

To determine the degree of hydrolysis and hydrolysis constant of \emph{Anilinium Hydrochloride} using Potentiometer.

\section{Apparatus Required}

\begin{itemize}
    \item Potentiometer
    \item Platinum Electrode
    \item Calomel Electrode
\end{itemize}

\section{Chemicals Required}

\begin{itemize}
    \item Anilinium Hydrochloride
    \item Quinhydrone
\end{itemize}

\section{Experimental Data}

\subsection*{Calculation of Hydrolysis Constant}

\begin{enumerate}
    \item \(pH\) is given by \[pH = \frac{- E_{\text{obs}} + E_{\text{QH}} + E_{\text{cal}}}{0.0591}\] where \(E_{\text{QH}} = 0.6996 \, V\) and \(E_{\text{cal}} = -0.242 \, V\).\footnote{These are oxidation potentials. We took the sign conventions accordingly.}
    \item Since \(pH = -\log[H^+] = -\log(c\alpha)\), \(pH = -\log(c) - \log(\alpha)\), the degree of hydrolysis \(\alpha\) can be calculated at any given concentration.
    \item From \(\alpha\), using \(K_h = \frac{c\alpha^2}{1-\alpha}\), we can deduce hydrolysis constant of Anilinium Hydrochloride.
    \item The dissociation constant can also be calculated using the relation \(K_b = \frac{K_w}{K_h}\).\footnote{The value of \(K_w\) at \(25^\circ\) C is assumed as \(10^{-14}\).}
\end{enumerate}

\begin{table}[H]
    \centering
    \begin{tabular}{|c|c|c|c|c|c|}
    \hline
    {[}\ch{C6H5NH}\(^{3+}\)\ch{Cl-}{]} &
      \(E_{\text{obs}}\) &
      \(pH\) &
      \(\alpha\) {[}\(\times 10^{-2}\){]} &
      \(K_h\) {[}\(\times 10^{-5}\){]} &
      \(K_b\) {[}\(\times 10^{-10}\){]} \\ \hline
    0.10 & 0.277 & 3.06 & 0.88 & 0.78 & 12.8 \\ \hline
    0.05 & 0.274 & 3.11 & 1.56 & 1.24 & 8.04 \\ \hline
    0.02 & 0.264 & 3.28 & 2.65 & 1.44 & 6.93 \\ \hline
    0.01 & 0.260 & 3.34 & 4.53 & 2.15 & 4.64 \\ \hline
    \end{tabular}
    \end{table}

\section{Conclusion}

The experimented value of \(K_h = 1.40 \times 10^{-5}\) and that of \(K_b = 8.11 \times 10^{-10}\).

\end{document}