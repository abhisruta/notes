\documentclass[11pt]{scrartcl}
\usepackage[sexy]{evan}
\usepackage[a4paper, portrait, margin=0.75in]{geometry}
\usepackage{cases}

\begin{document}
    
    \title{Naive Set Theory}
    \subtitle{
        \color{Brown}{Instructor: Saugata Bandopadhyay} \\ {\small \color{darkgray}{Department of Mathematics and Statistics, IISER Kolkata}}}
    \maketitle

    \vspace{-2em}

    \section{Notion of Sets}
    A \emph{set} is a collection of objects which are known as elements or members. Elements of a set can be anything, such as numbers, lines, students, fishes, and even sets.\footnote{Here we are not going to Axiomatic Definition of Sets, rather taking "Naive" approach. To be precise, ZFC axioms can eliminate paradoxes and ambiguities.}. Just as a box can be empty, a set can be empty as well. We denote \emph{empty set} as \(\phi\). A set containing single element called \emph{singleton set}.
    
    \begin{definition}
        We shall use uppercase letters to label sets and lowercase letters to label elements in a set. And, let \(A\) be set and \(x\) be an object. Then, we write: \begin{itemize}
            \item \(x \in A\) if \(x\) is an element of \(A\).
            \item \(x \notin A\) if \(x\) is not an element of \(A\).
        \end{itemize}
    \end{definition}

    \begin{remark}
        Note that, \(\phi \neq \left\{\phi\right\}\) and \(a \neq \left\{a\right\}\).
    \end{remark}

    \section{Basic Terminologies}
    Since a set is defined by its elements, we write a set by listing/defining its elements by two ways.
    \begin{enumerate}
        \item \textbf{Roster Method}
        \item \textbf{Set-Builder Method}
    \end{enumerate}
    \begin{definition}[Empty Set]
        The set that has no element is called empty set. We denote the empty set as \(\phi\). Furthermore, we call a set \(A\) non-empty, if \(A \neq \phi\), if A has atleast one elementin it.
    \end{definition}
    \begin{remark}
        Same as Remark 1.2.
    \end{remark}
    \begin{definition}[\vocab{Subset}]
        Let \(A, B\) be two sets. We say that \(A\) is a subset of \(B\), denoted by \(A \subseteq B\), if every element of \(A\) is also an element of \(B\).
    \end{definition}
    \begin{definition}[\vocab{Equality of Sets}]
        Let \(A, B\) be two sets. We say that \(A\) and \(B\) are equal, denoted by \(A = B\), if \(A \subseteq B\) and \(B \subseteq A\), i.e., every element of \(A\) is also an element of \(B\) and vice versa.
    \end{definition}
    \begin{definition}[\vocab{Proper Subset}]
        Let \(A, B\) be two sets. We say that \(A\) is a proper subset of \(B\), denoted by \(A \subset B\), if \(A \subseteq B\) and \(A \neq B\).
    \end{definition}

    \section{Operations on Sets}
    In this section, we introduce various set operations that allow us generate new sets.
    \begin{definition}
        Let \(A, B\) be two sets. Then, 
        \begin{itemize}
            \item \vocab{Union.} The union of \(A\) and \(B\), denoted by \(A \cup B\), is defined as: \[A \cup B \defeq \left\{x : x \in A \vee x \in B\right\}\]
            \item \vocab{Intersection.} The intersection of \(A\) and \(B\), denoted by \(A \cap B\), is defined as: \[A \cap B \defeq \left\{x : x \in A \wedge x \in B\right\}\]
            \item \vocab{Difference.} The intersection of \(A\) and \(B\), denoted by \(A \setminus B\), is defined as: \[A \setminus B \defeq \left\{x : x \in A \wedge x \notin B\right\}\]
            \item \vocab{Symmetric Difference.} The intersection of \(A\) and \(B\), denoted by \(A \triangle B\), is defined as: \[A \triangle B \defeq \left(A \setminus B\right) \cup \left(B \setminus A\right)\]
        \end{itemize}
    \end{definition}
    \begin{example}
        Let \(A = \left\{1, 2, 3, 4, 5\right\}\) and \(B = \left\{1, 4, 6, 9\right\}\). Then: 
        \begin{itemize}
            \item \(A \cup B = \left\{1, 2, 3, 4, 5, 6, 9\right\}\)
            \item \(A \cap B = \left\{1, 4\right\}\)
            \item \(A \setminus B = \left\{2, 3, 5\right\}\)
            \item \(B \setminus A = \left\{6, 9\right\}\)
            \item \(A \triangle B = \left\{2, 3, 5, 6, 9\right\}\)
        \end{itemize}
    \end{example}
    \begin{theorem}
        Let \(A, B\) and \(C\) be three sets. Then:
        \begin{enumerate}
            \item \emph{Associativity}: \(A \cup (B \cup C) = (A \cup B) \cup C\) and \(A \cap (B \cap C) = (A \cap B) \cap C\)
            \item \emph{Distributivity}: \(A \cup (B \cap C) = (A \cup B) \cap (A \cup C)\)
            \item \emph{Commutativity}: \(A \cup B = B \cup A\) and \(A \cap B = B \cap A\)
        \end{enumerate}
    \end{theorem}
    \begin{proof}
        Left as an exercise to the reader.
    \end{proof}
    \begin{definition}[\vocab{Disjoint Sets}]
        Let \(A, B\) be two sets. We say that \(A, B\) are disjoint if \(A \cap B = \phi\), i.e., \(A, B\) have no element in common.
    \end{definition}
    \begin{definition}[\vocab{Complement}]
        Let \(X\) be \emph{Universal Set} and let \(A \subseteq X\). The complement of \(A\) (relative to \(X\)) is defined as: \[A^c \defeq X \setminus A = \{x \in X : x \notin A\}\]
    \end{definition}
    \begin{remark}
        It is important to define the universal set unambiguously in order to be able to talk about the complement of a set.

        Also, note that, \((A^{c})^{c} = A\).
    \end{remark}
    \begin{theorem}[De Morgan's Law]
        Let \(X\) be the universal set and let \(A, B \subseteq X\). Then:
        \begin{enumerate}
            \item \((A \cup B)^{c} = A^c \cap B^c\)
            \item \((A \cap B)^c = A^c \cup B^c\)
        \end{enumerate} 
    \end{theorem}
    \begin{proof}
        \begin{enumerate}
            \item Let \(x \in (A \cup B)^c = X \setminus (A \cup B)\). Then, \(x \in X\) and \(x \notin A \cup B\). Therefore, \(x \notin A\) and \(x \notin B \implies x \in X \setminus A\) and \(x \in X \setminus B \implies x \in A^c\) and \(x \in B^c \implies x \in A^c \cap B^c \). Therefore, \((A \cup B)^c \subseteq A^c \cap B^c\).
            
            Conversely, let \(y \in A^c \cap B^c\). Then, \(y \in X\) and \(y \notin A\) and \(y \notin B\). Therefore, \(y \notin A \cup B\). Hence, \(y \in (A \cup B)^c \implies A^c \cap B^c \subseteq (A \cup B)^c\).

            By the definition of equality of set, we conclude, \[(A \cup B)^c = A^c \cap B^c\]
            
            \item Using (1), we have, \((A^c \cup B^c)^c = (A^c)^c \cap (B^c)^c \implies (A^c \cup B^c)^c = A \cap B \implies \left[(A^c \cup B^c)^c\right]^c\) \\ Which implies, \[A^c \cup B^c = (A \cap B)^c\]
        \end{enumerate}
    \end{proof}
    \begin{definition}[\vocab{Power Set}]
        Let \(A\) be a set. The \emph{power set of} \(A\), denoted by \(\PP(A)\) is the set of subsets of \(A\).
    \end{definition}
    \begin{remark}
        Notice that,
        \begin{itemize}
            \item The power set of \(A\) is also defined by \(2^A\).
            \item \(\PP(A) \neq \phi\) for any set \(A\), as \(phi \in \PP(A)\).
            \item Let \(n \in \NN\) and let \(A\) be set of \(n\) elements. Then, \(\PP(A)\) has exactly \(2^{n}\) elements.
        \end{itemize}
    \end{remark}
    \begin{example}
        Look at the examples as follows:
        \begin{itemize}
            \item Let \(A = \phi\). Then, \(\PP(A) = \{\phi\}\).
            \item Let \(A = \{\phi\}\). Then, \(\PP(A) = \{\phi, \{\phi\}\} = \{\phi, A\}\).
            \item Let \(A = \{a, b\}\). Then, \(\PP(A) = \{\phi, \{a\}, \{b\}, A\}\).
        \end{itemize}
    \end{example}
    We now introduce the notion of the cartesian product of two sets. Let us begin with the definition of an ordered pair.
    \begin{definition}[\vocab{Ordered Pair}, \emph{informal}]
        Let \(A, B\) be two non-empty sets and let \(a \in A, b \in B\). The \emph{ordered pair} \((a, b)\) is a notation specifying \(a\) and \(b\), in that \emph{order}.
    \end{definition}
    \begin{remark}
        The definition is unsatisfactory because it is only descriptibe and based on intuitive sense of order (we didn't define order rigorously).

        We shall now give a formal definition of the notion of an ordered pair that captures the main property of an ordered pair, which is described in Lemma 3.14.

        There are several ways to formally define ordered pair. We here state the one given by \emph{Kuratowski (1921)}.
    \end{remark}
    \begin{definition}[\vocab{Ordered Pair}, \emph{formal}]
        Let \(A, B\) be non-empty sets and let \(a \in A, b \in B\). The \emph{ordered pair} \((a,b)\) is defined as \[(a, b) \defeq \{\{a\}, \{a, b\}\}\]
    \end{definition}
    \begin{lemma}
        Let \(A, B\) be non-empty sets and let \(a, x \in A\) and \(b, y \in B\). Then: \[(a, b) = (x, y) \iff a = x \wedge b = y\] 
    \end{lemma}
    \begin{proof}
        Let us suppose that, \((a, b) = (x, y)\). We claim that \(a = x \wedge b = y\). We consider two cases.

        \begin{description}
            \item[Case I:] \(\mathbf{(a = b)}\) Then, \((a, b) = \{\{a\}, \{a, b\}\} = \{\{a\}, \{a\}\} = \{\{a\}\}\). \\ Therefore, \(\{x, y\} \in \{\{x\}, \{x, y\}\} = (x, y) = (a, b) = \{\{a\}\}\). \\ Which implies that, \(\{x, y\} = {a}\). Hence, \(x = y = a\), i.e., \(x = a \wedge y = a = b\).
            \item[Case II:] \(\mathbf{(a \neq b)}\) In this case, first, we prove the claim:
            \begin{claim}
                \(x \neq y\) for \(a \neq b\).
            \end{claim}
            \begin{proof}
                Suppose to the contrary, \(x = y\). Then, \(\{a, b\} \in (a, b) = (x, y) = \{\{x\}\}\). Which implies that \(a = b = x\), which contradicts the hypothesis. Therefore, \(x \neq y\).
            \end{proof}
            Next, as \(\{x\} \in (x, y) = (a, b) = \{\{a\}, \{a, b\}\}\), we have, \(\{x\} = \{a\}\) as \(\{x\} \neq \{a, b\}\) because \(a \neq b\). Hence, \(x = a\).

            It remains to show that \(y = b\). We have \(\{x, y\} \in (x, y) = (a, b)\). Therefore, \(\{x, y\} = \{a, b\}\) as, \(\{x, y\} \neq \{a\}\) because, \(x \neq y\). This implies that, \(y = b\) as \(y \neq a\) because \(x \neq y\).

            Therefore, \(x = a\) and \(y = b\).

            Conversely, it's trivial that if \(a = x \wedge b = y\), then \((a, b) = \{\{a\}, \{a, b\}\} = \{\{x\}, \{x, y\}\} = (x, y)\)
        \end{description}
    \end{proof}
    \begin{definition}[\vocab{Cartesian Product}]
        Let \(A, B\) be two sets. The \emph{cartesian product} of \(A\) and \(B\), denoted by \(A \times B\), is defined as: \[A \times B \defeq \begin{cases}
            \{(a, b) : a \in A, b \in B\} \quad \text{if} \quad A \neq \phi \wedge B \neq \phi \\
            \phi, \quad \text{otherwise}
        \end{cases}\]
    \end{definition}
    \begin{example}
        Look at the examples as follows:
        \begin{itemize}
            \item Let \(A = \{1, 2\}, B = \{a, b, c\}\). Then, \[A \times B = \{(1, a), (1, b), (1, c), (2, a), (2, b), (2, c)\}\]
            \item Let \(A = \{a, b\}\). Then,\[A \times A = \{(a, a), (a, b), (b, a), (b, b)\}\] 
        \end{itemize}
    \end{example}

\end{document}